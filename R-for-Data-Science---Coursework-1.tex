% Options for packages loaded elsewhere
\PassOptionsToPackage{unicode}{hyperref}
\PassOptionsToPackage{hyphens}{url}
%
\documentclass[
]{article}
\usepackage{lmodern}
\usepackage{amssymb,amsmath}
\usepackage{ifxetex,ifluatex}
\ifnum 0\ifxetex 1\fi\ifluatex 1\fi=0 % if pdftex
  \usepackage[T1]{fontenc}
  \usepackage[utf8]{inputenc}
  \usepackage{textcomp} % provide euro and other symbols
\else % if luatex or xetex
  \usepackage{unicode-math}
  \defaultfontfeatures{Scale=MatchLowercase}
  \defaultfontfeatures[\rmfamily]{Ligatures=TeX,Scale=1}
\fi
% Use upquote if available, for straight quotes in verbatim environments
\IfFileExists{upquote.sty}{\usepackage{upquote}}{}
\IfFileExists{microtype.sty}{% use microtype if available
  \usepackage[]{microtype}
  \UseMicrotypeSet[protrusion]{basicmath} % disable protrusion for tt fonts
}{}
\makeatletter
\@ifundefined{KOMAClassName}{% if non-KOMA class
  \IfFileExists{parskip.sty}{%
    \usepackage{parskip}
  }{% else
    \setlength{\parindent}{0pt}
    \setlength{\parskip}{6pt plus 2pt minus 1pt}}
}{% if KOMA class
  \KOMAoptions{parskip=half}}
\makeatother
\usepackage{xcolor}
\IfFileExists{xurl.sty}{\usepackage{xurl}}{} % add URL line breaks if available
\IfFileExists{bookmark.sty}{\usepackage{bookmark}}{\usepackage{hyperref}}
\hypersetup{
  pdftitle={GY7702 Assignment 1},
  pdfauthor={Ben Coombs},
  hidelinks,
  pdfcreator={LaTeX via pandoc}}
\urlstyle{same} % disable monospaced font for URLs
\usepackage[margin=1in]{geometry}
\usepackage{color}
\usepackage{fancyvrb}
\newcommand{\VerbBar}{|}
\newcommand{\VERB}{\Verb[commandchars=\\\{\}]}
\DefineVerbatimEnvironment{Highlighting}{Verbatim}{commandchars=\\\{\}}
% Add ',fontsize=\small' for more characters per line
\usepackage{framed}
\definecolor{shadecolor}{RGB}{248,248,248}
\newenvironment{Shaded}{\begin{snugshade}}{\end{snugshade}}
\newcommand{\AlertTok}[1]{\textcolor[rgb]{0.94,0.16,0.16}{#1}}
\newcommand{\AnnotationTok}[1]{\textcolor[rgb]{0.56,0.35,0.01}{\textbf{\textit{#1}}}}
\newcommand{\AttributeTok}[1]{\textcolor[rgb]{0.77,0.63,0.00}{#1}}
\newcommand{\BaseNTok}[1]{\textcolor[rgb]{0.00,0.00,0.81}{#1}}
\newcommand{\BuiltInTok}[1]{#1}
\newcommand{\CharTok}[1]{\textcolor[rgb]{0.31,0.60,0.02}{#1}}
\newcommand{\CommentTok}[1]{\textcolor[rgb]{0.56,0.35,0.01}{\textit{#1}}}
\newcommand{\CommentVarTok}[1]{\textcolor[rgb]{0.56,0.35,0.01}{\textbf{\textit{#1}}}}
\newcommand{\ConstantTok}[1]{\textcolor[rgb]{0.00,0.00,0.00}{#1}}
\newcommand{\ControlFlowTok}[1]{\textcolor[rgb]{0.13,0.29,0.53}{\textbf{#1}}}
\newcommand{\DataTypeTok}[1]{\textcolor[rgb]{0.13,0.29,0.53}{#1}}
\newcommand{\DecValTok}[1]{\textcolor[rgb]{0.00,0.00,0.81}{#1}}
\newcommand{\DocumentationTok}[1]{\textcolor[rgb]{0.56,0.35,0.01}{\textbf{\textit{#1}}}}
\newcommand{\ErrorTok}[1]{\textcolor[rgb]{0.64,0.00,0.00}{\textbf{#1}}}
\newcommand{\ExtensionTok}[1]{#1}
\newcommand{\FloatTok}[1]{\textcolor[rgb]{0.00,0.00,0.81}{#1}}
\newcommand{\FunctionTok}[1]{\textcolor[rgb]{0.00,0.00,0.00}{#1}}
\newcommand{\ImportTok}[1]{#1}
\newcommand{\InformationTok}[1]{\textcolor[rgb]{0.56,0.35,0.01}{\textbf{\textit{#1}}}}
\newcommand{\KeywordTok}[1]{\textcolor[rgb]{0.13,0.29,0.53}{\textbf{#1}}}
\newcommand{\NormalTok}[1]{#1}
\newcommand{\OperatorTok}[1]{\textcolor[rgb]{0.81,0.36,0.00}{\textbf{#1}}}
\newcommand{\OtherTok}[1]{\textcolor[rgb]{0.56,0.35,0.01}{#1}}
\newcommand{\PreprocessorTok}[1]{\textcolor[rgb]{0.56,0.35,0.01}{\textit{#1}}}
\newcommand{\RegionMarkerTok}[1]{#1}
\newcommand{\SpecialCharTok}[1]{\textcolor[rgb]{0.00,0.00,0.00}{#1}}
\newcommand{\SpecialStringTok}[1]{\textcolor[rgb]{0.31,0.60,0.02}{#1}}
\newcommand{\StringTok}[1]{\textcolor[rgb]{0.31,0.60,0.02}{#1}}
\newcommand{\VariableTok}[1]{\textcolor[rgb]{0.00,0.00,0.00}{#1}}
\newcommand{\VerbatimStringTok}[1]{\textcolor[rgb]{0.31,0.60,0.02}{#1}}
\newcommand{\WarningTok}[1]{\textcolor[rgb]{0.56,0.35,0.01}{\textbf{\textit{#1}}}}
\usepackage{longtable,booktabs}
% Correct order of tables after \paragraph or \subparagraph
\usepackage{etoolbox}
\makeatletter
\patchcmd\longtable{\par}{\if@noskipsec\mbox{}\fi\par}{}{}
\makeatother
% Allow footnotes in longtable head/foot
\IfFileExists{footnotehyper.sty}{\usepackage{footnotehyper}}{\usepackage{footnote}}
\makesavenoteenv{longtable}
\usepackage{graphicx,grffile}
\makeatletter
\def\maxwidth{\ifdim\Gin@nat@width>\linewidth\linewidth\else\Gin@nat@width\fi}
\def\maxheight{\ifdim\Gin@nat@height>\textheight\textheight\else\Gin@nat@height\fi}
\makeatother
% Scale images if necessary, so that they will not overflow the page
% margins by default, and it is still possible to overwrite the defaults
% using explicit options in \includegraphics[width, height, ...]{}
\setkeys{Gin}{width=\maxwidth,height=\maxheight,keepaspectratio}
% Set default figure placement to htbp
\makeatletter
\def\fps@figure{htbp}
\makeatother
\setlength{\emergencystretch}{3em} % prevent overfull lines
\providecommand{\tightlist}{%
  \setlength{\itemsep}{0pt}\setlength{\parskip}{0pt}}
\setcounter{secnumdepth}{-\maxdimen} % remove section numbering

\title{GY7702 Assignment 1}
\author{Ben Coombs}
\date{12/11/2020}

\begin{document}
\maketitle

\hypertarget{gy7702-assignment-1}{%
\section{\texorpdfstring{\textbf{GY7702 Assignment
1}}{GY7702 Assignment 1}}\label{gy7702-assignment-1}}

\hypertarget{loading-libraries}{%
\subsection{Loading libraries}\label{loading-libraries}}

Throughout this assignment, I will be using the libraries tidyverse and
knitr. Therefore it is a good idea to load them straight away.

\begin{Shaded}
\begin{Highlighting}[]
\KeywordTok{library}\NormalTok{(tidyverse)}
\KeywordTok{library}\NormalTok{(knitr)}
\end{Highlighting}
\end{Shaded}

\hypertarget{question-1}{%
\subsection{Question 1}\label{question-1}}

\hypertarget{q1.1}{%
\subsubsection{Q1.1}\label{q1.1}}

\begin{Shaded}
\begin{Highlighting}[]
\CommentTok{#Create the vector of 25 numbers listed on the question paper}
\NormalTok{nums <-}\StringTok{ }\KeywordTok{c}\NormalTok{(}\OtherTok{NA}\NormalTok{, }\DecValTok{3}\NormalTok{, }\DecValTok{4}\NormalTok{, }\DecValTok{4}\NormalTok{, }\DecValTok{5}\NormalTok{, }\DecValTok{2}\NormalTok{, }\DecValTok{4}\NormalTok{, }\OtherTok{NA}\NormalTok{, }\DecValTok{6}\NormalTok{, }\DecValTok{3}\NormalTok{, }\DecValTok{5}\NormalTok{, }\DecValTok{4}\NormalTok{, }\DecValTok{0}\NormalTok{, }\DecValTok{5}\NormalTok{, }\DecValTok{7}\NormalTok{, }\DecValTok{5}\NormalTok{, }\OtherTok{NA}\NormalTok{, }\DecValTok{5}\NormalTok{, }\DecValTok{2}\NormalTok{, }\DecValTok{4}\NormalTok{, }\OtherTok{NA}\NormalTok{, }
          \DecValTok{3}\NormalTok{, }\DecValTok{3}\NormalTok{, }\DecValTok{5}\NormalTok{, }\OtherTok{NA}\NormalTok{)}
\CommentTok{#Create a new vector of the same numbers, this time with missing values (NA }
\CommentTok{#values omitted)}
\NormalTok{nums_new <-}\StringTok{ }\NormalTok{nums[}\OperatorTok{!}\KeywordTok{is.na}\NormalTok{(nums)]}
\CommentTok{#Check if all responses are strongly agree or strongly disagree}
\KeywordTok{all}\NormalTok{(nums_new }\OperatorTok\StringTok{ }\KeywordTok{is.element}\NormalTok{(}\KeywordTok{c}\NormalTok{(}\DecValTok{1}\NormalTok{,}\DecValTok{7}\NormalTok{)))}
\end{Highlighting}
\end{Shaded}

\begin{verbatim}
## [1] FALSE
\end{verbatim}

A return of FALSE indicates that there were participants in the survey
that did not either completely agree or completely disagree.

\hypertarget{q1.2}{%
\subsubsection{Q1.2}\label{q1.2}}

\begin{Shaded}
\begin{Highlighting}[]
\CommentTok{#Return the positions in the vector of elements that are participants responding }
\CommentTok{#somehow agree or stronger (5 or greater).}
\KeywordTok{which}\NormalTok{(nums_new }\OperatorTok{>=}\StringTok{ }\DecValTok{5}\NormalTok{)}
\end{Highlighting}
\end{Shaded}

\begin{verbatim}
## [1]  4  7  9 12 13 14 15 20
\end{verbatim}

\hypertarget{question-2}{%
\subsection{Question 2}\label{question-2}}

\hypertarget{q2.1}{%
\subsubsection{Q2.1}\label{q2.1}}

\begin{Shaded}
\begin{Highlighting}[]
\CommentTok{# Install and load the library "palmerpenguins"}
\KeywordTok{library}\NormalTok{(palmerpenguins)}
\end{Highlighting}
\end{Shaded}

\hypertarget{q2.2}{%
\subsubsection{Q2.2}\label{q2.2}}

\begin{Shaded}
\begin{Highlighting}[]
\CommentTok{#Create a table to show species, island, bill length and body mass of the 10 }
\CommentTok{#Gentoo penguins with the highest body mass}
\NormalTok{gentoo_penguins <-}\StringTok{ }\NormalTok{penguins }\OperatorTok
\StringTok{  }\CommentTok{#Only want the table to show species, island, bill length and body mass}
\StringTok{  }\KeywordTok{select}\NormalTok{(species, island, bill_length_mm, body_mass_g) }\OperatorTok
\StringTok{  }\CommentTok{#Only display rows of Gentoo penguins}
\StringTok{  }\KeywordTok{filter}\NormalTok{(species }\OperatorTok{==}\StringTok{ "Gentoo"}\NormalTok{) }\OperatorTok
\StringTok{  }\CommentTok{#Arrange the rows by body mass largest to smallest}
\StringTok{  }\KeywordTok{arrange}\NormalTok{(}\OperatorTok{-}\NormalTok{body_mass_g) }\OperatorTok
\StringTok{  }\CommentTok{#Take the first 10 values, i.e. the 10 Gentoo penguins with the largest body }
\StringTok{  }\CommentTok{#mass}
\StringTok{  }\KeywordTok{slice_head}\NormalTok{(}\DataTypeTok{n =} \DecValTok{10}\NormalTok{)}

\CommentTok{# Display the table}
\NormalTok{knitr}\OperatorTok{::}\KeywordTok{kable}\NormalTok{(gentoo_penguins)}
\end{Highlighting}
\end{Shaded}

\begin{longtable}[]{@{}llrr@{}}
\toprule
species & island & bill\_length\_mm & body\_mass\_g\tabularnewline
\midrule
\endhead
Gentoo & Biscoe & 49.2 & 6300\tabularnewline
Gentoo & Biscoe & 59.6 & 6050\tabularnewline
Gentoo & Biscoe & 51.1 & 6000\tabularnewline
Gentoo & Biscoe & 48.8 & 6000\tabularnewline
Gentoo & Biscoe & 45.2 & 5950\tabularnewline
Gentoo & Biscoe & 49.8 & 5950\tabularnewline
Gentoo & Biscoe & 48.4 & 5850\tabularnewline
Gentoo & Biscoe & 49.3 & 5850\tabularnewline
Gentoo & Biscoe & 55.1 & 5850\tabularnewline
Gentoo & Biscoe & 49.5 & 5800\tabularnewline
\bottomrule
\end{longtable}

\hypertarget{q2.3}{%
\subsubsection{Q2.3}\label{q2.3}}

\begin{Shaded}
\begin{Highlighting}[]
\CommentTok{#Create a table of the average bill length per island ordered by average bill }
\CommentTok{#length}
\NormalTok{island_bill_length <-}\StringTok{ }\NormalTok{penguins }\OperatorTok
\StringTok{  }\CommentTok{#Only considering island and bill length}
\StringTok{  }\KeywordTok{select}\NormalTok{(island, bill_length_mm) }\OperatorTok
\StringTok{  }\CommentTok{#Remove NA values to correctly summarise}
\StringTok{  }\KeywordTok{filter}\NormalTok{(}\OperatorTok{!}\KeywordTok{is.na}\NormalTok{(bill_length_mm)) }\OperatorTok
\StringTok{  }\CommentTok{#Group rows by island}
\StringTok{  }\KeywordTok{group_by}\NormalTok{(island) }\OperatorTok
\StringTok{  }\CommentTok{#Summarise the islands by the average bill length}
\StringTok{  }\KeywordTok{summarise}\NormalTok{(}
    \DataTypeTok{avg_bill_length_mm =} \KeywordTok{mean}\NormalTok{(bill_length_mm)}
\NormalTok{  ) }\OperatorTok
\StringTok{  }\CommentTok{#Arrange the table by highest to lowest average bill length}
\StringTok{  }\KeywordTok{arrange}\NormalTok{(}\OperatorTok{-}\NormalTok{avg_bill_length_mm)}

\NormalTok{knitr}\OperatorTok{::}\KeywordTok{kable}\NormalTok{(island_bill_length)}
\end{Highlighting}
\end{Shaded}

\begin{longtable}[]{@{}lr@{}}
\toprule
island & avg\_bill\_length\_mm\tabularnewline
\midrule
\endhead
Biscoe & 45.25749\tabularnewline
Dream & 44.16774\tabularnewline
Torgersen & 38.95098\tabularnewline
\bottomrule
\end{longtable}

\end{document}
